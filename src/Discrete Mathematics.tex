\documentclass[a4paper, 12pt]{report}

\usepackage[dvipsnames]{xcolor}

%%%%%%%%%%%%%%%%
% Set Variables %
%%%%%%%%%%%%%%%%

\def\useItalian{0}  % 1 = Italian, 0 = English

\def\courseName{Discrete Mathematics}

\def\coursePrerequisites{
    \begin{itemize}
        \item Differential Calculus
        \item Integral Calculus
    \end{itemize}
}

\def\book{"TODO",\\Author TODO, ...}

% \def\authorName{Simone Bianco}
% \def\email{bianco.simone@outlook.it}
% \def\github{https://github.com/Exyss/university-notes}
% \def\linkedin{https://www.linkedin.com/in/simone-bianco}

\def\authorName{Alessio Bandiera}
\def\email{alessio.bandiera02@gmail.com}
\def\github{https://github.com/aflaag-notes}
\def\linkedin{https://www.linkedin.com/in/alessio-bandiera-a53767223}

% Do not change

%%%%%%%%%%%%
% Packages %
%%%%%%%%%%%%

\usepackage{../Nyx/nyx-packages}
\usepackage{../Nyx/nyx-styles}
\usepackage{../Nyx/nyx-frames}
\usepackage{../Nyx/nyx-macros}
\usepackage{../Nyx/nyx-title}
\usepackage{../Nyx/nyx-intro}

%%%%%%%%%%%%%%
% Title-page %
%%%%%%%%%%%%%%

\logo{../Nyx/logo.png}

\ifx\useItalian\1
    \institute{\curlyquotes{\hspace{0.25mm}Sapienza} Università di Roma}
    \faculty{Ingegneria dell'Informazione,\\Informatica e Statistica}
    \department{Dipartimento di Informatica}
    \subtitle{Appunti integrati con il libro \book}
    \author{\textit{Autore}\\\authorName}
\else
    \institute{\curlyquotes{\hspace{0.25mm}Sapienza} University of Rome}
    \faculty{Faculty of Information Engineering,\\Informatics and Statistics}
    \department{Department of Computer Science}
    \subtitle{Lecture notes integrated with the book \book}
    \author{\textit{Author}\\\authorName}
\fi

\title{\courseName}
\date{\today}

% \supervisor{Linus \textsc{Torvalds}}
% \context{Well, I was bored\ldots}

%%%%%%%%%%%%
% Document %
%%%%%%%%%%%%

\begin{document}
    \maketitle

    % The following style changes are valid only inside this scope 
    {
        \hypersetup{allcolors=black}
        \fancypagestyle{plain}{%
        \fancyhead{}        % clear all header fields
        \fancyfoot{}        % clear all header fields
        \fancyfoot[C]{\thepage}
        \renewcommand{\headrulewidth}{0pt}
        \renewcommand{\footrulewidth}{0pt}}

        \romantableofcontents
    }

    \introduction

    %%%%%%%%%%%%%%%%%%%%%

    \chapter{TODO}
    
    \section{Solved exercises}

    \subsection{Number theory}

    \begin{framedprob}{$n^2 + n$ is even}
        Show that for every $n \in \N$, $n^2 + n$ is an even number.
    \end{framedprob}

    \begin{proof}
        Note that $n^2 + n = n \cdot (n + 1)$, hence:

        \begin{itemize}
            \item if $n$ is even, then $$\exists k \in \N \mid n = 2k \implies n(n + 1) = 2k(2k + 1) =4k^2 + 2k = 2(k^2+k)$$ which is an even number;
            \item if $n$ is odd, then $$\exists k \in \N \mid n = 2k + 1 \implies n(n + 1) = (2k + 1)(2k + 2) = 4k^2 + 6k + 2 = 2(2k^2 + 3k +1)$$ which is an even number.
        \end{itemize}
    \end{proof}

    \begin{framedprob}{$4n -1$ is not prime}
        Show that there are infinitely many numbers of the form $4n -1$ that are not prime.
    \end{framedprob}
    
    \begin{proof}
        Note that $$\forall x^2 \in \N - \{0\} \quad 4x^2 - 1 = (2x +1)(2x -1)$$ which is a proper factorization of $4x^2 -1$, hence every perfect square yields a number of the form $4n -1$ which is not a prime number. Note that the number of perfect squares is infinite since the set of perfect square has the same cardinality of $\N$ since it's possibile to construct a bijective function as follows: $$\funcmap{f}{\N}{\N}{x}{x^2}$$

        Also, note that this proof does not show \tit{every non-prime number of the form $4n -1$}, since that is outside the scope of the problem.
    \end{proof}

    \begin{framedprob}{The $4n - 3$ set}
        Consider the following set: $$S := \{4n - 3 \mid n \in \N\}$$

        \begin{enumerate}
            \item Show that $S$ closed under multiplication.
            \item A number $p$ is said to be \tit{$S$-prime} if and only if $p$ is the product of exactly two factors of $S$; for example, even though $3 ^2 = 9 \notin \Primes$ we have that $9 = 1 \cdot 9$, and since $1 = 4 \cdot 1 - 3 \in S$ and $9 = 4 \cdot 3 - 3 \in S$, then $9$ is $S$-prime. Is the set of $S$-prime numbers infinite?
            \item TODO
        \end{enumerate}
    \end{framedprob}

    \begin{proof}
        \quad
        \begin{enumerate}
            \item To show that $S$ is closed under multiplication, it suffices to show that $$\forall a, b \in \N \quad (4a -3)(4b- 3) = 16ab -12a -12b +9 = 4(4ab - 3a -3b +3) - 3 \in S$$
            \item TODO
        \end{enumerate}
    \end{proof}

    \subsection{Induction}

    \begin{framedprob}{Cardinality of the power set}
        Show that for every given set $S$ such that $n := \abs{S}$ it holds that $\abs{\powerset(S)} = 2^n$.
    \end{framedprob}

    \proofind{
        The statement will be shown by induction over $n$, the number of elements contained into $S$.
    }{
        $n = 0 \implies S = \varnothing \implies \powerset(S) = \powerset(\varnothing) = \{\varnothing\} \implies \abs{\powerset(S)} = 1 = 2^0 = 2^n$.
    }{
        Assume that the statement is true for some fixed integer $n$.
    }{
        It must be shown that, for a given set of elements $S$ such that $\abs{S} = n+ 1$, it holds true that $\abs{\powerset(S)} = 2 ^{n +1}$. Consider a subset $S' \subseteq S$ such that $\abs{S'} = \abs{S} - 1 = n + 1 - n = n$, hence for the inductive hypothesis we have that $\abs{\powerset(S')} = 2^n$. Thus, to get the cardinality of $\powerset(S)$ the $(n + 1)$-th element inside $S - S'$ must be paired with every of the sets contained inside $\powerset(S')$, hence $$\powerset(S) = 2 \cdot \powerset(S') = 2 \cdot 2^n = 2^{n + 1}$$
    }

    \subsection{Continued fractions}

    \begin{framedprob}{Limits of continued fractions}
        \begin{enumerate}
            \item What is the value that the following limit approaches? $$\lim_{n \to + \infty}{[2; 1, 4, n]}$$
            \item Consider the following sequence: $$\dfrac{25}{16}, \dfrac{49}{36}, \dfrac{81}{64}, \dfrac{121}{100}, \ldots$$ Compute the continued fractions of these ratios; what is the limit of this sequence?
        \end{enumerate}
    \end{framedprob}

    \begin{proof}
        \quad
        \begin{enumerate}
            \item By using the CFA, we get the following table:

            \begin{center}
                \begin{tabular}{c|c|c|c|c|c} 
                    C.F. & & 2 & 1 & 4 & $n$ \\
                    \hline
                    $N$ & 1 & 2 & 3 & 14 & $14 \cdot n + 3$ \\
                    \hline
                    $D$ & 0 & 1 & 1 & 5 & $5 \cdot n + 1$ \\
                    \hline
                \end{tabular}
            \end{center}

            which means that $$[2; 1, 4, n] = \dfrac{14n + 3}{5n + 1} \implies \lim_{n \to + \infty}{\dfrac{14n + 3}{5n + 1}} = \dfrac{14}{5}$$
        \item We can convince ourselves that the sequence is $$\rbk{\dfrac{2k + 1}{2k}}^2$$ for some $k \in \N$. Thus we can compute the continued fractions of the given ratios (calculations omitted) and get the following results: $$\centeredsoe{k = 2 \implies \rbk{\dfrac{2 \cdot 2 + 1}{2 \cdot 2}}^2 = \rbk{\dfrac{5}{4}}^2 =  \dfrac{25}{16} = [1;1,1,3,2] \\k = 3 \implies \rbk{\dfrac{2 \cdot 3 + 1}{2 \cdot 3}}^2 = \rbk{\dfrac{7}{6}}^2 = \dfrac{49}{36} = [1;2,1,3,3] \\ k = 4 \implies \rbk{\dfrac{2 \cdot 4 + 1}{2 \cdot 4}}^2 = \rbk{\dfrac{9}{8}}^2 =\dfrac{81}{64} =[1;3,1,3,4] \\ k = 5 \implies \rbk{\dfrac{2 \cdot 5 + 1}{2 \cdot 5}}^2 = \rbk{\dfrac{11}{10}}^2 =\dfrac{121}{100} = [1;4,1,3,5]}$$ and we can easily prove that $$\rbk{\dfrac{2k + 1}{2k}}^2 = [1; k - 1, 1, 3, k]$$ by using the CFA and constructing the following table: 

            \begin{center}
                \begin{tabular}{c|c|c|c|c|c|c} 
                    C.F. & & 1 & $k - 1$ & 1 & 3 & $k$ \\
                    \hline
                    $N$ & 1 & 1 & $k$ & $k + 1$ & $4k +3$ & $4k^2 + 4k + 1$ \\
                    \hline
                    $D$ & 0 & 1 & $k - 1$ & $k$ & $4k - 1$ & $4k^2$ \\
                    \hline
                \end{tabular}
            \end{center}
            
            Ultimately, the limit approaches $$\lim_{k \to + \infty}{\dfrac{4k^2 + 4k + 1}{4k^2}} = \dfrac{4}{4} = 1$$
    \end{enumerate}
    \end{proof}

    \begin{framedprob}{Binomial coefficients}
        Prove that $$\forall p \in \Primes, k \in \N \mid p > k > 1 \quad \congmod{\binom{p}{k}}{0}{p}$$
    \end{framedprob}

    \begin{proof}
        Note that $$\binom{p}{k} = \dfrac{p!}{k! (p - k) !} = p \cdot \dfrac{(p - 1) !}{k ! (p - k) !} \implies p \bigg\vert \binom{p}{k}$$ and note that, since $p \in \Primes$, $p$ can't be simplified with any of the factors of the denominator (since $p > k$ and $p > p - k$ because $k > 1$). Hence $\congmod{\binom{p}{k}}{0}{p}$
    \end{proof}

    \begin{framedprob}{Systems of congruence equations}
        Solve the following system: $$\soe{l}{\congmod{x + 2y}{4}{7} \\ \congmod{4x+3y}{4}{7}}$$ Are there any solutions in $\Z_5$?
    \end{framedprob}

    \begin{proof}
        Note that $$x + 2y \equiv 4 \ (\bmod \ 7) \iff x = 4 - 2y \ (\bmod \ 7)$$ that we can substitute $x$ in the second equation as follows $$\centeredsoe{4 \cdot (4 - 2y)  + 3y \equiv 16 - 8 y  + 3y\equiv 2 - 5y \equiv 4 \ (\bmod \ 7) \iff \\ \iff \congmod{-5y}{2}{7} \iff \congmod{2y}{2}{7} \iff \congmod{y}{1}{7}}$$ and then $$\congmod{x + 2 \cdot 1}{4}{7} \iff \congmod{x}{2}{7}$$

        Instead, if we try to solve the following system $$\soe{l}{\congmod{x + 2y}{4}{5} \\ \congmod{4x+3y}{4}{5}}$$ and we substitute $x$ in the second equation, we get that $$16 - 8y + 3y \equiv 1 - 5y \ \equiv 4 \ (\bmod \ 5) \iff -5y \equiv 5 \ (\bmod \ 5)$$ but since $\gcd(-5, 5) = -5 \neq 1$ then $[5] \notin \Z_5^*$, which means that the system has no solutions.
    \end{proof}

    \begin{framedprob}{Quadratic congruence equations}
        Solve the following equation in $\Z_{11}$ $$x^2 + 3x + 4 \equiv 0 \ (\bmod \ 11)$$
    \end{framedprob}

    \begin{proof}
        By solving for $x$ in $\Z_{11}$ we get that \centeredeq{0.9}{$x_{1, 2} \equiv \dfrac{-3 \pm \sqrt{9 - 4 \cdot 4}}{2} \equiv \dfrac{-3 \pm \sqrt{-7}}{2} \equiv \dfrac{-3 \pm \sqrt 4}{2} \equiv \dfrac{-3 \pm 2}{2} \equiv \dfrac{8 \pm 2}{2} \implies \soe{l}{\congmod{x}{5}{11} \\ \congmod{x}{3}{11}}$}
    \end{proof}
\end{document}
