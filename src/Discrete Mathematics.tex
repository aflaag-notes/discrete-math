\documentclass[a4paper, 12pt]{report}

\usepackage[dvipsnames]{xcolor}

%%%%%%%%%%%%%%%%
% Set Variables %
%%%%%%%%%%%%%%%%

\def\useItalian{0}  % 1 = Italian, 0 = English

\def\courseName{Discrete Math}

\def\coursePrerequisites{TODO: idk}

\def\book{"TODO",\\Author TODO, ...}

% \def\authorName{Simone Bianco}
% \def\email{bianco.simone@outlook.it}
% \def\github{https://github.com/Exyss/university-notes}
% \def\linkedin{https://www.linkedin.com/in/simone-bianco}

\def\authorName{Alessio Bandiera}
\def\email{alessio.bandiera02@gmail.com}
\def\github{https://github.com/aflaag-notes}
\def\linkedin{https://www.linkedin.com/in/alessio-bandiera-a53767223}

% Do not change

%%%%%%%%%%%%
% Packages %
%%%%%%%%%%%%

\usepackage{../Nyx/nyx-packages}
\usepackage{../Nyx/nyx-styles}
\usepackage{../Nyx/nyx-frames}
\usepackage{../Nyx/nyx-macros}
\usepackage{../Nyx/nyx-title}
\usepackage{../Nyx/nyx-intro}

%%%%%%%%%%%%%%
% Title-page %
%%%%%%%%%%%%%%

\logo{../Nyx/logo.png}

\ifx\useItalian\1
    \institute{\curlyquotes{\hspace{0.25mm}Sapienza} Università di Roma}
    \faculty{Ingegneria dell'Informazione,\\Informatica e Statistica}
    \department{Dipartimento di Informatica}
    \subtitle{Appunti integrati con il libro \book}
    \author{\textit{Autore}\\\authorName}
\else
    \institute{\curlyquotes{\hspace{0.25mm}Sapienza} University of Rome}
    \faculty{Faculty of Information Engineering,\\Informatics and Statistics}
    \department{Department of Computer Science}
    \subtitle{Lecture notes integrated with the book \book}
    \author{\textit{Author}\\\authorName}
\fi

\title{\courseName}
\date{\today}

% \supervisor{Linus \textsc{Torvalds}}
% \context{Well, I was bored\ldots}

%%%%%%%%%%%%
% Document %
%%%%%%%%%%%%

\begin{document}
    \maketitle

    % The following style changes are valid only inside this scope 
    {
        \hypersetup{allcolors=black}
        \fancypagestyle{plain}{%
        \fancyhead{}        % clear all header fields
        \fancyfoot{}        % clear all header fields
        \fancyfoot[C]{\thepage}
        \renewcommand{\headrulewidth}{0pt}
        \renewcommand{\footrulewidth}{0pt}}

        \romantableofcontents
    }

    \introduction

    %%%%%%%%%%%%%%%%%%%%%

    \chapter{Number Theory}
    
    \section{TODO}
    
    \subsection{TODO}

    \begin{frameddefn}{Peano's axioms}
        The \tbf{Peano's axioms} are 5 axioms which define the set $\N$ of the \tbf{natural numbers}, and they are the following:

        \begin{enumerate}[label=\roman*), font=\itshape]
            \item $0 \in \N$
            \item $\exists \func{\succfn}{\N}{\N}$, or equivalently, $\forall x \in \N \quad \succfn(x) \in \N$
            \item $\forall x, y \in \N \quad x \neq y \implies \succfn(x) \neq \succfn(y)$
            \item $\nexists x \in \N \mid \succfn(x) = 0$
            \item $\forall S \subseteq \N \quad (0 \in S \land (\forall x \in S \quad \succfn(x) \in S)) \implies S = \N$
        \end{enumerate}

        \imp{Note} inside this notes, it will be assumed that $0 \in \N$.
    \end{frameddefn}

    \begin{framedprinc}{Induction principle}
        Let $P$ be a property which is true for $n = 0$, thus $P(0)$ is true; also, for every $n \in \N$ we have that $P(n) \implies P(n + 1)$; then $P(n)$ is true for every $n \in \N$.

        Using symbols, using the formal logic notation, we have that $$\dfrac{P(0) \quad P(n) \implies P(n + 1)}{ \forall n \quad P(n )}$$
    \end{framedprinc}

    \begin{framedobs}{The fifth Peano's axiom}
        Note that the fifth Peano's axiom is equivalent to the induction principle, since, it states that for every subset $S$ of $\N$ containing 0 and closed under $\succfn$ must be equal to $\N$ itself.
    \end{framedobs}

    \begin{frameddefn}{Integers}
        TODO
    \end{frameddefn}

    \begin{frameddefn}{Divisor}
        TODO
    \end{frameddefn}

    \begin{example}[Divisors]
        TODO
    \end{example}

    \begin{frameddefn}[label={primes}]{$\Primes$}
        TODO
    \end{frameddefn}

    \begin{framedprop}[label={primes inf}]{$\Primes$ is infinite}
        There are infinitely many primes. Using symbols $$\abs{\Primes} = + \infty$$
    \end{framedprop}

    \begin{proof}
        By way of contradiction, assume that $\Primes$ is finite, thus $$\exists n \in \N \mid \Primes = \{p_1, \ldots, p_n\}$$ and let $x = p_1 \cdot \ldots \cdot p_n$. Since $x \neq p_1, \ldots, p_n$, then $x \notin \Primes$, so $x$ is not a prime number; but $x$ can't be divided by any of the $p_1, \ldots, p_n$ either, because the remainder will always be 1. This means that $x$ is neither prime nor non-prime, which is a contradiction $\lightning$.
    \end{proof}

    \begin{frameddefn}[label={gcd}]{$\gcd$}
        The $\tbf{gcd}$ (\tit{Greatest Common Divisor}) of two given numbers $a$, $b$ is the greatest of the divisors which $a$ and $b$ have in common. Using symbols, we say that $$d = \gcd(a, b) \iff \forall f \in \N : f \mid a \land f \mid b \quad f \mid d$$ If the $\gcd$ of two numbers is 1, they are said to be \tbf{coprime}.
    \end{frameddefn}

    \begin{example}[$\gcd$]
        Given 15 and 63, we have that $\gcd(15, 63) = 3$.
    \end{example}

    \begin{framedalgo}[label={euclid}]{Euclid's algorithm}
        \textbf{Input}: Two natural numbers $a, b$.\\
        \textbf{Output}: $\gcd(a, b)$.

        \hrule
        \begin{algorithmic}[1]
            \Function{gcd}{$a, b$}
                \State TODO
                % \State $r_0 := b$
                % \State $r_1 := a$
                % \State $r_{i - 1} := r_1$
            \EndFunction
        \end{algorithmic}
    \end{framedalgo}

    \begin{example}[Euclid's algorithm]
        \label{euclid example}
        To compute the $\gcd(341, 527)$, using the \cref{euclid}, we get the following: $$\centeredsoe{527 = 341 \cdot 1 + 186 \\ 341 = 186 \cdot 1 + 155 \\ 186 = 155 \cdot 1 + 31 \\ 155 = 31 \cdot 5 + 0}$$ hence we have that $$\gcd(341, 527) = 31$$
    \end{example}

    \begin{framedlem}[label={bezout}]{Bézout's identity}
        Given a pair of numbers $a, b \in \Z$, there exists $x, y \in \Z$ such that the $\gcd(a, b)$ is a \href{https://en.wikipedia.org/wiki/Linear_combination}{linear combination} of $a$ and $b$. Using symbols $$\forall a, b \in \Z \quad \exists x, y \in \Z \mid \gcd(a, b) = ax + by$$
    \end{framedlem}

    \begin{proof}
        Omitted.
    \end{proof}

    \begin{example}[Bézout's identities]
        Using the \cref{euclid example}, in order to compute the Bézout's identity of 341 and 527, we need to do the following: $$31 = 186 - 155 \cdot 1 = 186 - (341 - 186 \cdot 1) = 2 \cdot 186 - 341 = 2 \cdot (527 - 341) - 341 = 2 \cdot 527 - 3 \cdot 341$$ thus the Bézout's identity is $$31 = 2 \cdot 527 - 3\cdot 341$$
    \end{example}

    \begin{framedcor}[label={prime divisors cor}]{Prime divisors}
        Given a natural number $n \in \N$ and a prime number $p \in \Primes$ , it holds true that $$p \nmid n \iff \gcd(p, n) = 1$$
    \end{framedcor}

    \proofiff{
        Instead of proving that $p \nmid n \implies \gcd(p, n) = 1$, we will prove the contrapositive, namely that $\gcd(p, n) > 1 \implies p \mid n$. Hence, since $\gcd(p, n) \mid p$ by definition, because $p \in \Primes$ then $\gcd(p, n)$ must be either 1 or $p$ itself, and we assumed that $\gcd(p, n) > 1$, $\gcd(p, n)$ must be 1, which means that $p \mid n$.
    }{
        Note that $\gcd(p, n) = 1 \implies \exists x, y \in \Z \mid 1 = px + ny$ by the \cref{bezout}, hence if $p \mid a$ then $p \mid 1$ by the \cref{gcd}, which is impossibile because $p \in \Primes$ by the \cref{primes}.
    }

    \begin{framedlem}{Prime divisors}
        Given a pair of numbers $a, b \in \N$, and a prime number $p \in \Primes$ such that $p \mid ab$, then either $p \mid a$ or $p \mid b$. Using symbols $$\forall a, b \in \N \quad \exists p \in \Primes : p \mid ab \implies p \mid a \lor p \mid b$$
    \end{framedlem}

    \begin{proof}
        Without loss of generality, assume that $p \nmid a$, thus $\gcd(p, a) = 1$ by the \cref{prime divisors cor}; hence, for the \cref{bezout}, we have that $$\exists x, y \in \Z \mid 1 = px + ay \iff b = bpx + bay$$ Note that $p \mid ab \iff \exists k \in \Z \mid pk = ab$ which means that $$b = bpx + pky = p(bx+ky) \iff p \mid b$$ The same argument can be used to show that $p \nmid b \implies p \mid a$.
    \end{proof}

    \begin{framedthm}[label={upf}]{Fundamental theorem of arithmetic}
        The \tbf{fundamental theorem of arithmetic}, also known as the \tbf{UPF} theorem (\tit{Unique Prime Factorization}) states that for every natural number $n \in \N$ there exists a unique prime factorization for $n$. Using symbols $$\forall n \in \N \quad \exists ! p_1, \ldots, p_k \in \Primes, e_1, \ldots, e_k \in \N \mid n = {p_1}^{e_1} \cdot \ldots \cdot {p_k}^{e_k}$$
    \end{framedthm}

    \begin{proof}
        Omitted.
    \end{proof}

    \section{Solved exercises}

    \subsection{TODO}

    \begin{framedprob}{$n^2 + n$ is even}
        Show that for every $n \in \N$, $n^2 + n$ is an even number.
    \end{framedprob}

    \begin{proof}
        Note that $n^2 + n = n \cdot (n + 1)$, hence:

        \begin{itemize}
            \item if $n$ is even, then $$\exists k \in \N \mid n = 2k \implies n(n + 1) = 2k(2k + 1) =4k^2 + 2k = 2(k^2+k)$$ which is an even number;
            \item if $n$ is odd, then $$\exists k \in \N \mid n = 2k + 1 \implies n(n + 1) = (2k + 1)(2k + 2) = 4k^2 + 6k + 2 = 2(2k^2 + 3k +1)$$ which is an even number.
        \end{itemize}
    \end{proof}

    \begin{framedprob}{$4n -1$ is not prime}
        Show that there are infinitely many numbers of the form $4n -1$ that are not prime.
    \end{framedprob}
    
    \begin{proof}
        Note that $$\forall x^2 \in \N - \{0\} \quad 4x^2 - 1 = (2x +1)(2x -1)$$ which is a proper factorization of $4x^2 -1$, hence every perfect square yields a number of the form $4n -1$ which is not a prime number. Note that the number of perfect squares is infinite since the set of perfect square has the same cardinality of $\N$ since it's possibile to construct a bijective function as follows: $$\funcmap{f}{\N}{\N}{x}{x^2}$$

        Also, note that this proof does not show \tit{every non-prime number of the form $4n -1$}, since that is outside the scope of the problem.
    \end{proof}

    \subsection{Induction}

    \begin{framedprob}{Cardinality of the power set}
        Show that for every given set $S$ such that $n := \abs{S}$ it holds true that $\abs{\powerset(S)} = 2^n$.
    \end{framedprob}

    \proofind{
        The statement will be shown by induction over $n$, the number of elements contained into $S$.
    }{
        $n = 0 \implies S = \varnothing \implies \powerset(S) = \powerset(\varnothing) = \{\varnothing\} \implies \abs{\powerset(S)} = 1 = 2^0 = 2^n$.
    }{
        Assume that the statement is true for some fixed integer $n$.
    }{
        It must be shown that, for a given set of elements $S$ such that $\abs{S} = n+ 1$, it holds true that $\abs{\powerset(S)} = 2 ^{n +1}$. Consider a subset $S' \subseteq S$ such that $\abs{S'} = \abs{S} - 1 = n + 1 - n = n$, hence for the inductive hypothesis we have that $\abs{\powerset(S')} = 2^n$. Thus, to get the cardinality of $\powerset(S)$ the $(n + 1)$-th element inside $S - S'$ must be paired with every of the sets contained inside $\powerset(S')$, hence $$\powerset(S) = 2 \cdot \powerset(S') = 2 \cdot 2^n = 2^{n + 1}$$
    }

    \begin{framedprob}{The $4n - 3$ set}
        Consider the following set: $$S := \{4n - 3 \mid n \in \N\}$$

        \begin{enumerate}
            \item Show that $S$ closed under multiplication.
            \item A number $p$ is said to be \tit{$S$-prime} if and only if $p$ is the product of exactly two factors of $S$; for example, even though $3 ^2 = 9 \notin \Primes$ we have that $9 = 1 \cdot 9$, and since $1 = 4 \cdot 1 - 3 \in S$ and $9 = 4 \cdot 3 - 3 \in S$, then $9$ is $S$-prime. Is the set of $S$-prime numbers infinite?
            \item TODO
        \end{enumerate}
    \end{framedprob}

    \begin{proof}
        \quad
        \begin{enumerate}
            \item To show that $S$ is closed under multiplication, it suffices to show that $$\forall a, b \in \N \quad (4a -3)(4b- 3) = 16ab -12a -12b +9 = 4(4ab - 3a -3b +3) - 3 \in S$$
            \item TODO
        \end{enumerate}
    \end{proof}

    \subsection{Series}

    \begin{frameddefn}{Series}
        TODO scrivi che possono convergere divergere etc
    \end{frameddefn}

    \begin{frameddefn}{The harmonic series}
        The harmonic series is defined as follows: $$\sum_{k = 1}^{+ \infty}{\dfrac{1}{k}} = 1 + \dfrac{1}{2} + \dfrac{1}{3}$$
    \end{frameddefn}

    \begin{framedprop}{Divergence of the harmonic series}
        The harmonic series diverges.
    \end{framedprop}

    \begin{proof}
        Suppose that the harmonic series converges, thus $$\exists S \mid \sum_{k = 1}^{+ \infty}{\dfrac{1}{k}} = S$$ then we have that $$\centeredsoe{S = 1 + \dfrac{1}{2} + \dfrac{1}{3} + \dfrac{1}{4} + \dfrac{1}{5} + \dfrac{1}{6} \ldots = \\ = \rbk{1 + \dfrac{1}{2}} + \rbk{\dfrac{1}{3} + \dfrac{1}{4}} + \rbk{\dfrac{1}{5} + \dfrac{1}{6}} + \ldots > \\ > \rbk{\dfrac{1}{2} + \dfrac{1}{2}} + \rbk{\dfrac{1}{4} + \dfrac{1}{4}} + \rbk{\dfrac{1}{6} + \dfrac{1}{6}} + \ldots = \\ = 1 + \dfrac{1}{2} + \dfrac{1}{3} + \ldots = S}$$ implying that $S > S \lightning$.
    \end{proof}

    \begin{frameddefn}[label={NON LO SO CAMBIAMI}]{TODO}
        TODO
    \end{frameddefn}

    \begin{frameddefn}{Reciprocal of primes}
        The sum of the reciprocal of the prime numbers diverges. Using symbols $$\sum_{p \in \Primes}{\dfrac{1}{p}} = + \infty$$
    \end{frameddefn}

    \begin{proof}
        Consider the following inequality: $$\forall n \in \N \quad \prod_{p \in \Primes \mid p \le n}{\dfrac{p}{p - 1}} > \sum_{k = 1}^{n}{\dfrac{1}{k}}$$ We can prove it with as follows:

        \begin{itemize}
            \item for any given $p \in \Primes$, the fraction $\dfrac{p}{p - 1}$ can be rewritten as follows, by using the MISSINGCREF: $$\forall p \in \Primes \mid p \leq n \quad \dfrac{p}{p - 1} = \dfrac{1}{\frac{p - 1}{p}} = \dfrac{1}{1 - \dfrac{1}{p}} = \sum_{k = 0}^{+ \infty}{\dfrac{1}{p^k}} = 1 + \dfrac{1}{p} + \dfrac{1}{p^2} + \ldots$$ which is the infinite sum of the reciprocal of the powers of some prime number $p$
            \item this means that $$\exists p_1, \ldots, p_j \in \Primes \mid \prod_{p \in \Primes \mid p \le n}{\dfrac{p}{p - 1}} = p_1 \cdot \ldots \cdot p_j  \implies \prod_{p \in \Primes \mid p \le n}{\dfrac{p}{p - 1}} =  \sum_{k = 0}^{+ \infty}{\dfrac{1}{p_1^k}} \cdot \ldots \cdot  \sum_{k = 0}^{+ \infty}{\dfrac{1}{p_j^k}}$$
            \item thus, thanks to the \cref{upf} this product expands to the sum of the reciprocal of every natural number that contains $p_1, \ldots, p_j$ in his prime factorization, namely $$\exists e_1, \ldots, e_j \in \N \mid\sum_{k = 0}^{+ \infty}{\dfrac{1}{p_1^k}} \cdot \ldots \cdot  \sum_{k = 0}^{+ \infty}{\dfrac{1}{p_j^k}} = \sum_{k = 0}^{+ \infty}{\dfrac{1}{p_1^{e_1} \cdot \ldots \cdot p_j^{e_j}}}$$
            \item finally, since $p_1, \ldots, p_j <= n$ this summation must contain at least every term contained inside $\displaystyle \sum_{k = 1}^{n}{\dfrac{1}{k}}$, which proves the inequality.
        \end{itemize}
    \end{proof}

    Now consider the following: $$\centeredsoe{\displaystyle \sum_{k = 1}^{+ \infty}{\dfrac{1}{k}} < \prod_{p \in \Primes \mid p \le n}{\dfrac{p}{p - 1}} \iff \\ \displaystyle \iff \log \rbk{\sum_{k = 1}^{+ \infty}{\dfrac{1}{k}}} < \rbk{\prod_{p \in \Primes \mid p \le n}{\dfrac{p}{p - 1}}} = \\ \displaystyle = \sum_{p \in \Primes \mid p \le n}{\log \rbk{\dfrac{p}{p - 1}}} = \sum_{p \in \Primes \mid p \le n}{\rbk{\log p - \log (p - 1)}}= \sum_{p \in \Primes \mid p \le n}{\int_{p - 1}^{p}{\dfrac{1}{x} \diff x}}}$$ and consider the area under the curve $\dfrac{1}{x}$ within the $[p - 1, p]$ interval, for some prime number $p$:

    TODO METTI GRAFICO

    since $$\forall x_1, x_2 \in \R \quad x_1 < x_2 \iff \dfrac{1}{x_1} > \dfrac{1}{x_2}$$ the function $\dfrac{1}{x}$ is monotonically decreasing, and in particular $$\forall p \in \Primes \mid p \le n \quad p -1 < p \implies \dfrac{1}{p - 1} > \dfrac{1}{p}$$ whic implies that the area under the curve $\dfrac{1}{x}$ within the $\dfrac[p-1, p]$ must be smaller than the area of the rectangle that has a base of of $p - (p - 1) = p - p + 1 = 1$ and an height of $\dfrac{1}{p - 1}$, namely an area of $1 \cdot \dfrac{1}{p - 1} = \dfrac{1}{1 - p}$. This implies that $$\sum_{p \in \Primes \mid p \le n}{\int_{p - 1}^p{\dfrac{1}{x} \diff x}} < \sum_{p \in \Primes \mid p \le n}{\dfrac{1}{p - 1}} < \sum_{p \in \Primes \mid p \le n}{TODO}$$

\end{document}
